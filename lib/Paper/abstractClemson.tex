\documentclass[11pt]{amsart}
\usepackage{graphicx,xspace}
\usepackage{fullpage}
\usepackage{hyperref}
\usepackage{amsmath}
\usepackage{amsfonts}
\usepackage{verbatim}

\newenvironment{definition}[1][Definition]{\begin{trivlist}
\item[\hskip \labelsep {\bfseries #1}]}{\end{trivlist}}
\newenvironment{example}[1][Example]{\begin{trivlist}
\item[\hskip \labelsep {\bfseries #1}]}{\end{trivlist}}



\title{Algebraic Framework for Discrete Models in Systems Biology} 
\author{Franziska Hinkelmann}

\begin{document}
\maketitle
\begin{abstract}
	
	Discrete models are increasingly used in systems biology. They can reveal
	valuable insight about the qualitative behavior of a system when it is
	infeasible to estimate enough parameters accurately to build a continuous
	model. Many discrete models can be formulated as polynomial dynamical systems,
	that is state and time discrete dynamical systems described by polynomials
	over a finite field. This provides access to the algorithmic theory of
	computational algebra and the theoretical foundation of algebraic geometry, 
	which helps with all aspects of the modeling process: construction,
	  analysis, and usage of the model to generate new biological hypotheses. 
	
	Simulation is a common practice for analyzing
	discrete models, but many systems are far too large to capture all the
	relevant dynamical features through simulation alone.  
	Converted to a polynomial dynamical systems, we can apply tools from computational algebra to analyze their dynamics. 
	 The key feature of biological systems that is exploited by our algorithms is their sparsity: 
	while the number of nodes in a biological network may be quite large, each node is affected only 
	by a small number of other nodes. In our experience with models arising in systems biology,
	 this structure leads to fast computations when using algebraic models, and thus efficient analysis.
	

\end{abstract}

\end{document}
