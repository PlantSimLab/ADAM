\documentclass[11pt]{amsart}
\usepackage{graphicx,xspace}
\usepackage{fullpage}
\usepackage{hyperref}
\usepackage{amsmath}
\usepackage{amsfonts}
\usepackage{verbatim}
 
\newenvironment{definition}[1][Definition]{\begin{trivlist}
\item[\hskip \labelsep {\bfseries #1}]}{\end{trivlist}}
\newenvironment{example}[1][Example]{\begin{trivlist}
\item[\hskip \labelsep {\bfseries #1}]}{\end{trivlist}}
 
 
 
\title{ADAM: Analysis of Discrete Models of Biological
Systems Using Computer Algebra}
\author{Franziska Hinkelmann$^{a,b}$,
Madison Brandon$^{c,*}$,
Bonny Guang$^{d,*}$,
Rustin McNeill$^{e,*}$, \\
Grigoriy Blekherman$^{a}$,
Alan Veliz-Cuba$^{a,b}$,
Reinhard Laubenbacher$^{a,b}$}
 
\begin{document}
\maketitle
{\footnotesize
        \centerline{$^a$Virginia Bioinformatics Institute, Blacksburg, VA 24061-0123, USA}
}
 
{\footnotesize
        \centerline{$^b$Department of Mathematics,
         Virginia Polytechnic Institute and State University, Blacksburg, VA 24061-0123, USA}
}
 
{\footnotesize
        \centerline{$^c$University of Tennessee - Knoxville, Knoxville, TN 37996-2513, USA}
}
 
{\footnotesize
        \centerline{$^d$Harvey Mudd College, Claremont, CA 91711-5901, USA}
}
 
 
{\footnotesize
        \centerline{$^e$University of North Carolina - Greensboro, Greensboro, NC 27402-6170, USA}
}
 
{\footnotesize
        \centerline{$^*$These authors contributed equally}
}
 
%%%%%%%%%%%%%%%%%%%%%%%%%%%%%%%%%%%%%%%%%%%%%%%%%%%%%%%%%%%%%%%%%%
% should we remove all we/our from the abstract? F
% proteins or genes
\begin{abstract}
{\bf Motivation:} Many biological systems are modeled qualitatively with discrete models, such as
probabilistic Boolean networks, logical models, bounded petri-nets, and agent-based models.
Simulation is a common practice for analyzing discrete models, but many systems are far too large
to capture all the relevant dynamical features through simulation alone. \\
{\bf Results:} We convert discrete models into algebraic models
and apply tools from computational algebra to analyze the dynamics of discrete systems.
The key feature of biological systems that is exploited by our algorithms is their sparsity: while the number of genes (or agents) in a
biological network may be quite large, each gene is affected only by a small number of other genes. This allows for fast Gr\"{o}bner basis
computations in the algebraic models, and thus efficient analysis.\\
{\bf Availability:} All algorithms and methods are available in our package Analysis of Dynamic
Algebraic Models (ADAM), a `modeler friendly' web-interface
that allows for fast analysis of large models, without requiring understanding of the underlying mathematics or any software installation. ADAM is available as a web tool, so it runs platform independent on all systems.\\
{\bf Contact:} \href{mailto:reinhard@vbi.vt.edu}{reinhard@vbi.vt.edu}\\
\end{abstract}
 
 
%%%%%%%%%%%%%%%%%%%%%%%%%%%%%%%%%%%%%%%%%%%%%%%%%%%%%%%%%%%%%%%%%%
\section{Introduction}
 
% Different discrete models and lack of analysis tools
Mathematical modeling is a crucial tool in understanding the dynamic behavior of complex
biological systems. With increasing amounts of data available, the need for software tools to analyze this data is growing. There are several requirements for such software. The main point being, that the software helps to answer questions the biologist is interested in, provided data that can actually be obtained. The software should be platform independent, so that all members of a research team can use it independent of their workstation. The tool should be suited for a large class of models to make it worth to learn its interface and be compatible with other modeling tools. In addition, the software should be actively updated and maintained.
 
Discrete models are widely used in studying dynamical behavior of biological systems. Model types include
(probabilistic) Boolean networks, logical networks, Petri-nets, Cellular
automata, and Agent-based (individual-based) models.
However, there is a lack of tools
for analyzing dynamical behavior of large-scale discrete models. Commonly, the analysis is performed by simulation, meaning that an
initial configuration of the system is iterated a prescribed number of times, or until a
steady state is found. Since the number of initial configurations grows exponentially in the number of variables, simulation
will not provide a complete picture of dynamical behavior for large models.
Simulation is especially inefficient for non-deterministic
systems, since simulations must be run many times to obtain meaningful
probability estimates.
 
% Math framework and Groebner basis
All of the different types of discrete models mentioned above can be
converted into the unified framework of polynomial dynamical systems
\cite{Alan:Bioinf2010, Hinkelmann:2010}. This allows us to apply tools from
computational commutative algebra to analyze their dynamics. It can also help to foster new collaborations of researchers from different backgrounds.
 
Here, we present the online software package ADAM, Analysis of Dynamic Algebraic Models \cite{ADAM}, an analysis tool to study the dynamics of a wide range of discrete models. ADAM is the successor to DVD, Discrete Visualizer of Dynamics \cite{DVD}, a tool to visualize the temporal evolution of small polynomial dynamical systems.
General features of ADAM are discussed briefly, and new features are explained more in detail. ADAM also contains a model repository with several models from publications available in ADAM specific format to be run and modified by the interested reader.
 
%%%%%%%%%%%%%%%%%%
\section{General Features}
% ADAM available for free, platform independent
We developed an
online software package ADAM, Analysis of Dynamic Algebraic Models \cite{ADAM}, which automatically converts discrete models into polynomial dynamical
systems, and then analyzes their dynamics by using various computational algebra techniques. Even for large systems, ADAM
computes key dynamic features, such as fixed points, in a matter of seconds.
ADAM is available online free of charge. It is platform
independent and does not require installation of any software or computer
algebra tool.
 
%% different input types
ADAM can analyze different discrete models. It supports the following {\bf inputs}.
\begin{itemize}
 \item Logical models generated with GINsim \cite{GINsim}
 \item Petri-nets generated with Snoopy \cite{Snoopy}
 \item polynomial dynamical systems
 \item Boolean networks
 \item probabilistic polynomial dynamical systems \cite{shmulevich}.
\end{itemize}
 
ADAM analyzes the provided model for dynamic features. This means, the temporal evolution of the system is investigated. ADAM finds stable attractors of a systems. These are either fixed points, i.e., one state that does not change over time once this state is reached, or limit cycles of, a collection of states between which the systems oscillates. The temporal evolution of the model can be visualized by the {\it phase space}, a graph of all possible states and their transitions. For small enough models, ADAM generates a graph of the complete phase space.
When the phase space is too large to be drawn or when a visualization is not
needed, ADAM can calculate all fixed points or limit cycles of a given length. Furthermore, the evolution of a single state can be determined by computing the trajectory initiating from this state and evaluating it until an attractor is reached. All of the above described features can be computed assuming synchronous updates, or an update schedule specified by the user. When calculating fixed points, it is irrelevant, whether synchronous or asynchronous updates are used, this is relevant in particular for logical models that are defined as asynchronous systems. 
 
For probabilistic networks, i.e., models where each variable has several choices of local update rules, ADAM can generate a graph of all possible updates, this means that states in the phase space have out degree $\geq 1$, since there are several transitions possible. ADAM can find all true fixed points, i.e., states where any combination of possible update rules transitions to the same state. 
 
Independent of network size, ADAM generates a {\it wiring diagram}. Wiring diagrams, also known as dependency graphs, showthe static relationship between the variables. All edges in ADAM’s wiring diagrams are functional edges, that is for some configuration of the system, a change in the input variable causes a change in the output variable (see Appendix \ref{sec:func} for more details). This means that ADAM determines all non-functional edges, which is oftentimes of interest to members of the logical models community.
 
For Boolean networks, ADAM calculates all functional circuits (see Appendix \ref{sec:func}). For a certain class of Boolean networks, namely conjunctive/disjunctive networks, ADAM computes a complete description of the phase space (see Appendix \ref{sec:conj}). 
 
Summarized, ADAM can generate the following {\bf output}.
\begin{itemize}
 \item a graph of wiring diagram
 \item fixed points (for deterministic and probabilistic systems)
 \item limit cycles of length $m$
 \item trajectories originating from a given initial state until a stable
 attractor is found
 \item dynamics for synchronous or asynchronous updates
 \item functional circuits for Boolean networks
 \item a complete description of the phase space for conjunctive/disjunctive
 networks.
\end{itemize}
 
%%%%%%%%%%%%%%%%%%%%%%%%%%%%%%%%%%%%%%%%%%%%%%%%%%%%%%%%%%%%%%%%%%
\section{Methods}
Logical models, Petri-nets, and Boolean networks are automatically converted
into the corresponding polynomial dynamical system as described in
\cite{Alan:Bioinf2010}, so that algorithms from computational
algebra can be used to analyze the dynamics.
 
We developed and implemented several different algorithms that allow one to analyze
important features of algebraic models when they are too large for pure simulation.
Most algorithms rely on Gr\"obner basis calculations to find key dynamic
features.
Since the polynomials in the algebraic
models originate from biological systems, we can exploit their structural
features to secure very fast Gr\"obner basis computations.
 
%explain that everything is a polynomial and how equations relate to dynamics
The key idea behind our algorithms is that discrete models have finitely many states and computations
can be performed over a finite field \cite{Alan:Bioinf2010,
Hinkelmann:2010}. Since any function over a finite field is a polynomial
\cite{Lidl:1997} we convert discrete models into polynomial dynamical systems
and use commutative algebra algorithms. More specifically, the problem of finding fixed points and limit cycles
can now be reformulated as solving a system of polynomial equations. We use Gr\"{o}bner basis techniques to solve the
resulting system of polynomial equations. Gr\"obner basis calculation is for polynomial systems what
Gauss-Jordan elimination is for linear systems: a structured way to transform
the original system to triangular shape without changing its solution space.
The triangular shape of the resulting systems allows for stepwise retrieval of the solutions of the system.
 
% Sparse systems
The efficiency of the Gr\"obner basis calculations is largely dependent on the
assumption that most discrete models arising from biological systems are
sparse, meaning that every variable is only affected by a small subset of the
total variables in the system. It has been suggested, that in robust gene
regulatory networks genes are regulated by only a handful of regulators
\cite{Leclerc:2008}. Thus the PDSs representing such biological networks are
sparse, i.e., each function depends only on a small subset of the total nodes.
 
% heuristic explanation for why we think the GB algorithm is fast on
%sparse systems..
In the worst case, computing Gr\"obner bases for a set of polynomials has a
complexity of doubly exponential in the number of solutions to the system.
However, in practice, Gr\"{o}bner bases are computable in a reasonable time, and
for the sparse systems over a finite field that are common in discrete
biological models, it is actually fairly fast.
%In the first place, computing
%Gr\"{o}bner bases in modular form have been shown to be much faster usually
%\cite{Brown:1971}.
Working with sparse polynomials means that simpler
S-polynomials, usually of short length, will be added, causing the Groebner
basis computation to be easier.
Based on benchmarking tests for
25 logical models of biological systems \cite{GINsim}
and randomly generated systems,
the computations are very fast, and finish on the scale of
seconds.
 
\section{Application} \label{benchmarks}
% Example application
We demonstrate the power of ADAM on a well understood model of the expression
pattern of the segment polarity genes in Drosophila melanogaster. \cite{AO} build a model for embryonic pattern formation in the fruit fly Drosophila melanogaster. The Boolean
model presented in \cite{AO} consists of 60 variables, resulting in a phase
space with over $10^{18}$ states. To analyze the model, one needs to rename the variables the Boolean rules given in \cite{AO} to $x1$ to $x60$. Then one can use ADAM to analyze the model: The model type is {\it PDS}, the number of states is $2$, i.e., present or absent. One can either pick {\it Boolean}, and enter the Boolean rules in the text-area or upload a text file with the Boolean rules. Alternatively, one can first convert the Boolean rules to polynomials over $\mathbb F_2$, and enter the polynomials with the choice {\it Polynomial}. The model file in ADAM format can be accessed at \cite{DrosophilaModel}. The rules in the model file have already been translated to {\it Polynomial} form. Once the model file is uploaded, one needs to chose the {\it Analysis} type. Since the model with $60$ variables is too large for simulation, we chose {\it Algorithm}. This means that instead of exhaustive enumeration of the state space, analysis of the dynamics is done via computer algebra by solving specific systems of equations. In {\it Options}, we set {\it Limit cycle length} to $1$, we are looking for fixed points, i.e., states that transition to themselves after one update iteration. We chose {\it synchronous} as updating scheme, as in \cite{AO} all variables are updated at the same time. Once all these choices have been made, we obtain the fixed points by clicking {\it Analyze}.
ADAM returns a link to the graph of the {\it dependency graph} or wiring diagram, the static relation of the different variables. Below, ADAM returns the number of fixed points and the fixed points themselves. Each row in the table corresponds to a stable attractor. Attractors are written as binary strings, e.g., $$000000001001101000000001001101100000001001101100000001001101.$$
This corresponds to the configuration of the system, with first 8 proteins and genes absent, the next one present, the next two absent, etc. In fact, this is exactly the steady state obtained in \cite[Figure 4(b)]{AO} when starting the system with an initial state representing the experimental observations of stage 8 embryos. The output is shown in Fig \ref{fig:alg}.
 
\begin{figure}[htb]
 \centering
 \includegraphics[width=0.95\textwidth]{DroAlgOutput.jpg}
 \caption{ADAM: Analysis of steady states of Drosophila model}
 \label{fig:alg}
\end{figure}
 
ADAM can also generate trajectories for a given initial state. For example, we
can pick the initial state that was used in \cite[Figure 4(a)]{AO}. Again, we
enter {\it PDS} witih $2$ as the number of states and upload the polynomials
describing the model. Instead of {\it Algorithms}, we now choose {\it
Simulation}. We are not interested in the number of fixed points of the
complete phase space, but in a single trajectory originating from a specific
initial state, so we choose {\it One trajectory} as simulation option. As
initial state we enter the one corresponding to \cite[Figure 4(a)]{AO},
$$100000010001000110000010001000000101000000000000000010001000.$$
%1 0 0 0 0 0 0 1 0 0 0 1 0 0 0 1 1 0 0 0 0 0 1 0 0 0 1 0 0 0 0 0 0 1 0 1 0 0 0 0 0 0 0 0 0 0 0 0 0 0 0 0 1 0 0 0 1 0 0 0
By clicking {\it Analyze}, we obtain the temporal evolution of this particular
state until it reaches a steady state. As predicted in \cite{AO}, the this
steady state is the fixed point from the previous paragraph, see Fig.
\ref{fig:traj}.

\begin{figure}[htb]
 \centering
 \includegraphics[width=0.95\textwidth]{DroTraj.jpg}
 \caption{ADAM: Trajectory of Drosophila model}
 \label{fig:traj}
\end{figure}

 
 
To summarize, ADAM correctly identifies the fixed points
in less than one second. All steady states have been determined previously in \cite{AO} by error prone manual investigation of the system. In \cite{AO}, the model is formulated as a set of Boolean rules. In order to determine the steady states, the system of Boolean expression was solved manually.
 
In addition, ADAM computes that are there no limit
cycles of length two or three. The model has not been analyzed previously for
limit cycles. The absence of two- and three cycles strengthens confidence in
the model, since oscillatory behavior has not been observed experimentally.
The model file in ADAM format can be accessed at \cite{DrosophilaModel}.
 
\subsection{Benchmark Calculations}
We analyzed logical models
available in the GINsim model repository \cite{GINsimRepo} as of August 2010. The
repository consists of models in GINsim XML format previously published in
peer reviewed journals. We converted all but two models into polynomial
dynamics systems. For these 27 models we computed the fixed points. Almost all
calculations finished in less than a second. The two largest networks,
consisting of over $10^{30}$ states, took around 20 minutes each, see
Figure \ref{fig:chart}.
 
In addition to the published models in \cite{GINsimRepo}, we analyzed
randomly generated networks
that have the same sparse structure that we
expect from biological systems. We tested a total of 50 networks with
50-100 nodes ($10^{15} - 10^{30}$ states) and up to 2 inputs per variables. The
fixed point calculations took less than half a second for each network on
a 2.7 GHz computer.
\begin{figure}[htb]
 \centering
 \includegraphics[width=0.95\textwidth]{GINSimChart.png}
 \caption{Runtime of fixed points calculations of several logical models from
 \cite{GINsimRepo}. Executed on a 2.7 GHz computer.}
 \label{fig:chart}
\end{figure}
 
 
%%%%%%%%%%%%%%%%%%%%%%%%%%%%%%%%%%%%%%%%%%%%%%%%%%%%%%%%%%%%%%%%%%
\section{Architecture}
ADAM is available as an online-tool. This spares the user the obstacle of installing software. ADAM's user interface is implemented in HTML. We use JavaScript to generate a dynamic website that changes as the user makes his choices. This simplifies the process of entering a model. For example, when defining the model type, i.e., Polynomial Dynamical System, Probabilistic Network, Petri-net, and Logical Model the next line asks for the number of states, $k$-bound, or nothing, appropriately. Input can be entered directly into the text-area on the form, or uploaded as a text document.
 
All mathematical algorithms are programmed in Macaulay2 \cite{M2}. Macaulay2 is a powerful computer algebra system. The routines for which fast execution is crucial are implemented in C/C++ as part of the Macaulay2 core.
Logical Models and petri-nets in XML format are parsed with Ruby's XmlSimple library. The interplay between HTML and Macaulay2 is also programmed in Ruby.
 
Output graphs are generated with Graphviz's {\it dot} command. When simulation is chosen as analysis method, Graphviz's {\it ccomps - connected components filter for graphs} is used to count the connected components. A Perl script is used to execute the Graphviz commands.
 
%%%%%%%%%%%%%%%%%%%%%%%%%%%%%%%%%%%%%%%%%%%%%%%%%%%%%%%%%%%%%%%%%%
\section{Community Integration}
In order to be compatible with other modeling software, ADAM can analyze logical models in GINsim XML format \cite{GINsim}, as well as petri-nets generated with Snoopy \cite{Snoopy}, also as XML model file. This allows modelers from the Logical models or petri-net community to benefit from ADAM's powerful analysis methods. Files in these XML formats can be directly uploaded to the website and then be simulated or analyzed as polynomial dynamical systems. Conversion happens automatically, as a result, the user is given a list of variable names (as specified in the XML file), and the corresponding $x_i$ used in the polynomial dynamical system, so that the output can easily be interpreted.
 
ADAM generates the graph of the wiring diagram, and for small enough models, a graph of the dynamic evolution, the phase space. Also, trajectories for specific initial states can be visualized. The user can choose between several output formats (gif, jpeg, pdf, and ps), so that the generated pictures can be used without further editing within several different software.
 
A model repository is part of the ADAM website. The repository consists of a collection of several previously published models in ADAM format. The models are extracted from publications, and rewritten in ADAM specific format, i.e., all variables are renamed to $x_i$ and the update rules from the original publication are reformulated as Boolean rules or polynomials. By experimenting with an existing model, new users can quickly understand the main functionality of ADAM. In addition to the model itself, the database entries contain a short summary of the biological system and relevant graphs, together with an analysis of dynamic features determined by ADAM and their biological explanation. The repository is work in progress and growing quickly, as researchers from several institutions are participating and  generating more entries for the repository.
 
%%% This section could use more work and might be placed more prominently at the 
% beginning
Because of their intuitive nature, discrete models are an excellent introduction to mathematical modeling for students of the life sciences. ADAM's model repository is a great starting point to familiarize students with the abstraction of discrete models such as Boolean networks.
 
%%%%%%%%%%%%%%%%%%%%%%%%%%%%%%%%%%%%%%%%%%%%%%%%%%%%%%%%%%%%%%%%%%
\section{Conclusion}
 
Discrete Modeling techniques are a useful tool for analyzing biological
systems. Upon translating a discrete model, such as logical networks,
petri-nets, or agent-based models into an algebraic model, rich mathematical
theory becomes available. This enables one to
avoid simulation which is limited because of combinatorial explosion. The algorithms
we developed are fast for sparse systems, a structure maintained by most biological
systems. All algorithms have been included in the software package ADAM\cite{ADAM},
which is user-friendly and available as a free web service.
ADAM is highly suitable to be used in a class room as a first
introduction to discrete models as it does not require the students to run
anything else but a web browser.
We hope to expand ADAM to an all-encompassing Discrete Toolkit which incorporates more
analytical methods, better visualization, and automatic conversion for more model types.
We also hope to analyze controlled algebraic models and expand theory to stochastic systems.
 
%%%%%%%%%%%%%%%%%%%%%%%%%%%%%%%%%%%%%%%%%%%%%%%%%%%%%%%%%%%%%%%%%%
\appendix
 
\section{Definitions}
 
\subsection{Polynomial Dynamical Systems}
To be self-contained, we briefly explain PDS and their key features.
Several types of discrete models, including Logical models, Petri
Nets, and Agent-Based models, can be translated into PDS \cite{Alan:Bioinf2010,Hinkelmann:2010}.
 
\subsubsection{Polynomial Dynamical System (PDS)}
A {\bf polynomial dynamical system} \cite{JLSS} over a finite field $k$ is a function
$$f = (f_1, \ldots, f_n) : k^n \rightarrow k^n,$$
with coordinate functions $f_i \in k[x_1, \ldots , x_n]$. Iteration of $f$ results
in a time-discrete dynamical system. PDSs are special cases of finite
dynamical systems, which are maps $X^n \rightarrow   X^n$ over arbitrary
finite sets $X$.
 
PDS have several dynamic features of biological
relevance. These include the number of components, component sizes, fixed
points, limit cycles, and limit cycle lengths.
\begin{example}
Let $k= \mathbb F_2$ and $f = (f_1, f_2, f_3) : \mathbb F_2^3 \rightarrow
\mathbb F_2^3$ with
\begin{align*}
f_1 &= x_1x_2x_3+x_1x_2+x_2x_3+x_2 \\
f_2 &= x_1x_2x_3+x_1x_2+x_1x_3+x_1+x_2 \\
f_3 &= x_1x_2x_3+x_1x_3+x_2x_3+x_1+x_2.
\end{align*}
The wiring diagram of $f$, which shows the static interaction of the three
variables, is
depicted in Figure \ref{fig:ex} (left) along with its phase space in Figure
~\ref{fig:ex} (right).
The phase space shows the temporal evolution of the systems. Each state is
represented as a vector of the values of the three variables $(x_1, x_2,
x_3)$.
The PDS described by $f$ has
two stable attractors: a fixed points, $(000)$, and a limit cycle of length
three, consisting of the states $(010)$, $(111)$, and $(011)$.
\end{example}
 
\begin{figure}[ht]
 \centering
 \includegraphics[scale=0.55]{exampleWD.jpg}
 \includegraphics[scale=0.55]{exampleSS.jpg}
 \caption{(left)
 Wiring diagram: static relationship between variables
 (right)
 Phase space: temporal evolution of the system
 }
 \label{fig:ex}
\end{figure}
 
\subsubsection{Probabilistic Polynomial Dynamical System}
A {\bf probabilistic PDS} over a finite field $k$ is a collection of functions
$$f = (\{f_{1,1}, \ldots, f_{1, r_1}\}, \ldots, \{f_{n, 1}, \ldots, f_{n, r_n}
\}) : k^n \rightarrow k^n,$$
together with a probability distribution for every coordinate that assigns the
probability that a specific function is chosen to update that coordinate.
The coordinate functions $f_{i,j}$ are in $k[x_1, \ldots , x_n]$.
Probabilistic PDS, specifically Boolean probabilistic networks (PBN), have been studied
extensively in \cite{shmulevich}.
 
ADAM analyzes probabilistic PDS. It can simulate the
complete phase space for small enough models, by generating every possible
transition and labeling the edge with its probability according to the
distribution. If no distribution is given, ADAM assumes a uniform distribution
on all functions. For large networks, ADAM's {\it Algorithm} choice computes
fixed points of probabilistic networks.
 
 
%%%%%%%%%%%
\subsubsection{Functional Edges} \label{sec:func}
An edge in the wiring diagram from $x_i$ to $x_j$ is considered
functional, if there exists a state $\hat x = (\hat x_1,  \ldots, \hat x_n)$ such
that $f_j( \hat x_1,  \ldots, a, \ldots \hat x_n) \neq f_j(\hat x_1, \ldots, b, \ldots
\hat x_n)$, where $a$ and $b$ are values for $x_i$, in other words, if there
is at least one state, such that changing only $x_i$ but keeping all other
values fixed, changes the next state of $x_j$.
In ADAM, all edges in the wiring diagram are functional.
 
For Boolean networks, ADAM identifies all functional circuits. A circuit is a
closed directed path in the wiring diagram and it is functional, if all its
edges are functional. For further discussion of
functional circuits, see \cite{Chaouiya}.
 
 
 
%%%%%%%%%%%%%%%%%%%%%%%%%%%%%%%%%%%%%%%%%%%%%%%%%%%%%%%%%%%%%%%%%%
\section{Algorithms}
 
\subsection{Analysis of stable attractors}
Every attractor in a PDS is either a
fixed point or a limit cycle. For small models, ADAM determines the complete
phase space by enumeration, for large models, ADAM computes fixed points and
limit cycles of a given length.
 
A state is a fixed point, if it transitions to itself after one update of the
system. A state is part of a limit cycle of length $m$, if,
after $m$ updates, it results in itself. Any fixed points of a PDS satisfies
the equation $f(x) = x$, as no coordinate of $x$ is changing as it is updated.
Similarly, states of a
limit cycle of length $m$ satisfy the equation $f^m(x) = x$. ADAM computes all
fixed points by solving the system $f_i(x) - x_i = 0$ for $i \in \{1, \ldots,
n\}$ simultaneously. To efficiently solve the resulting systems of polynomial
equations, we first compute the Gr\"obner
basis in lexicographic order for the ideal generated by the equations.
By the elimination and extension theorem \cite{IVA}, choosing a lexicographic order
allows to easily obtain the solutions.
We use the Gr\"obner basis calculations distributed with Macaulay2 \cite{M2}, a
computer algebra system, and found that for quotient rings over a finite field
the implementation `Sugarless' is more efficient than the default algorithm
with `Sugar' \cite{Sugar:1991}.
For limit cycles of length $m$, the solutions of $f^m(x)=x$ are found and then
grouped into cycles, by applying $f$ to each of the solutions.
 
%%%%%%%%%%%
\subsection{Conjunctive/Disjunctive Networks} \label{sec:conj}
Some classes of networks have a certain structure that can be
exploited to achieve faster calculations. In \cite{conjunctive}, Jarrah et al.
show that for conjunctive (disjunctive) networks key dynamic features can be found with
almost no computational effort. Conjunctive (resp disjunctive) networks consist of
functions using only the AND (resp. OR) operator.
We include a separate algorithm to analyze
dynamics in the case of conjunctive/disjunctive networks as described in
\cite{conjunctive}. Currently,
this option only works on strongly connected graphs
 
 
%%%%%%%%%%%%%%%%%%%%%%%%%%%%%%%%%%%%%%%%%%%%%%%%%%%%%%%%%%%%%%%%%%
\section*{Acknowledgments}
The authors would like to thank Prof Monika Heiner for clarifying the Petri
net terminology. 
Funding provided through NSF \#0755322
\begin{comment}, Army research grant, Alan's
NSF fund, other funds?
All students/mentors involved in Model repository!
\end{comment}
\bibliographystyle{plain}
%\bibliographystyle{authordate1}
\bibliography{ADAM}
\end{document}
