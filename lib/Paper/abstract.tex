\documentclass[11pt]{amsart}
\usepackage{graphicx,xspace}
\usepackage{fullpage}
\usepackage{hyperref}
\usepackage{amsmath}
\usepackage{amsfonts}
\usepackage{verbatim}

\newenvironment{definition}[1][Definition]{\begin{trivlist}
\item[\hskip \labelsep {\bfseries #1}]}{\end{trivlist}}
\newenvironment{example}[1][Example]{\begin{trivlist}
\item[\hskip \labelsep {\bfseries #1}]}{\end{trivlist}}



\title{Analysis of Discrete Models of Biological 
Systems Using Computer Algebra} 
\author{Franziska Hinkelmann}

\begin{document}
\maketitle
\begin{abstract}
Many biological systems are modeled qualitatively with discrete models,
such as probabilistic Boolean networks, logical models, bounded Petri nets,
and agent-based models.  Simulation is a common practice for analyzing
discrete models, but many systems are far too large to capture all the
relevant dynamical features through simulation alone.  
We convert discrete models into algebraic models and apply tools from computational algebra to analyze their dynamics. 
 The key feature of biological systems that is exploited by our algorithms is their sparsity: 
while the number of nodes in a biological network may be quite large, each node is affected only 
by a small number of other nodes. In our experience with models arising in systems biology and random models,
 this structure leads to fast computations when using algebraic models, and thus efficient analysis.  
All algorithms and methods are available in our package Analysis of Dynamic
Algebraic Models (ADAM), a user friendly web-interface that allows for fast
analysis of large models, without requiring understanding of the underlying
mathematics or any software installation. ADAM is available as a web tool,
so it runs platform independent on all systems.
\end{abstract}

\end{document}
