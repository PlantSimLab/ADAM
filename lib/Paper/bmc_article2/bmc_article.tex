%% BioMed_Central_Tex_Template_v1.05
%%                                      %
%  bmc_article.tex            ver: 1.05 %
%                                       %


%%%%%%%%%%%%%%%%%%%%%%%%%%%%%%%%%%%%%%%%%
%%                                     %%
%%  LaTeX template for BioMed Central  %%
%%     journal article submissions     %%
%%                                     %%
%%         <27 January 2006>           %%
%%                                     %%
%%                                     %%
%% Uses:                               %%
%% cite.sty, url.sty, bmc_article.cls  %%
%% ifthen.sty. multicol.sty		       %%
%%									   %%
%%                                     %%
%%%%%%%%%%%%%%%%%%%%%%%%%%%%%%%%%%%%%%%%%


%%%%%%%%%%%%%%%%%%%%%%%%%%%%%%%%%%%%%%%%%%%%%%%%%%%%%%%%%%%%%%%%%%%%%
%%                                                                 %%	
%% For instructions on how to fill out this Tex template           %%
%% document please refer to Readme.pdf and the instructions for    %%
%% authors page on the biomed central website                      %%
%% http://www.biomedcentral.com/info/authors/                      %%
%%                                                                 %%
%% Please do not use \input{...} to include other tex files.       %%
%% Submit your LaTeX manuscript as one .tex document.              %%
%%                                                                 %%
%% All additional figures and files should be attached             %%
%% separately and not embedded in the \TeX\ document itself.       %%
%%                                                                 %%
%% BioMed Central currently use the MikTex distribution of         %%
%% TeX for Windows) of TeX and LaTeX.  This is available from      %%
%% http://www.miktex.org                                           %%
%%                                                                 %%
%%%%%%%%%%%%%%%%%%%%%%%%%%%%%%%%%%%%%%%%%%%%%%%%%%%%%%%%%%%%%%%%%%%%%


\NeedsTeXFormat{LaTeX2e}[1995/12/01]
\documentclass[10pt]{bmc_article}    



% Load packages
\usepackage{cite} % Make references as [1-4], not [1,2,3,4]
\usepackage{url}  % Formatting web addresses  
\usepackage{ifthen}  % Conditional 
\usepackage{multicol}   %Columns
\usepackage[utf8]{inputenc} %unicode support
\usepackage{hyperref}
\usepackage{amsfonts}
\usepackage{amsmath}
%\usepackage{graphicx}
\usepackage{multirow} 
\usepackage{rotating}

%\usepackage{color}

%\usepackage{soul}
\newcommand{\hl}{\hi}
\newcommand{\ul}{\hi}

%\usepackage{ulem}

%\usepackage[applemac]{inputenc} %applemac support if unicode package fails
%\usepackage[latin1]{inputenc} %UNIX support if unicode package fails
\urlstyle{rm}
 
 
%%%%%%%%%%%%%%%%%%%%%%%%%%%%%%%%%%%%%%%%%%%%%%%%%	
%%                                             %%
%%  If you wish to display your graphics for   %%
%%  your own use using includegraphic or       %%
%%  includegraphics, then comment out the      %%
%%  following two lines of code.               %%   
%%  NB: These line *must* be included when     %%
%%  submitting to BMC.                         %% 
%%  All figure files must be submitted as      %%
%%  separate graphics through the BMC          %%
%%  submission process, not included in the    %% 
%%  submitted article.                         %% 
%%                                             %%
%%%%%%%%%%%%%%%%%%%%%%%%%%%%%%%%%%%%%%%%%%%%%%%%%                     


\def\includegraphic[#1]{}
%\def\includegraphics[#1]{}



\setlength{\topmargin}{0.0cm}
\setlength{\textheight}{21.5cm}
\setlength{\oddsidemargin}{0cm} 
\setlength{\textwidth}{16.5cm}
\setlength{\columnsep}{0.6cm}

\newboolean{publ}

%%%%%%%%%%%%%%%%%%%%%%%%%%%%%%%%%%%%%%%%%%%%%%%%%%
%%                                              %%
%% You may change the following style settings  %%
%% Should you wish to format your article       %%
%% in a publication style for printing out and  %%
%% sharing with colleagues, but ensure that     %%
%% before submitting to BMC that the style is   %%
%% returned to the Review style setting.        %%
%%                                              %%
%%%%%%%%%%%%%%%%%%%%%%%%%%%%%%%%%%%%%%%%%%%%%%%%%%
 

%Review style settings
%\newenvironment{bmcformat}{\begin{raggedright}\baselineskip20pt\sloppy\setboolean{publ}{false}}{\end{raggedright}\baselineskip20pt\sloppy}

%Publication style settings
\newenvironment{bmcformat}{\fussy\setboolean{publ}{true}}{\fussy}




% Begin ...
\begin{document}
\begin{bmcformat}


%%%%%%%%%%%%%%%%%%%%%%%%%%%%%%%%%%%%%%%%%%%%%%
%%                                          %%
%% Enter the title of your article here     %%
%%                                          %%
%%%%%%%%%%%%%%%%%%%%%%%%%%%%%%%%%%%%%%%%%%%%%%

\newenvironment{example}[1][Example]{\begin{trivlist}
\item[\hskip \labelsep {\bfseries #1}]}{\end{trivlist}}

\title{ADAM: Analysis of Discrete Models of Biological
Systems Using Computer Algebra}
 
%%%%%%%%%%%%%%%%%%%%%%%%%%%%%%%%%%%%%%%%%%%%%%
%%                                          %%
%% Enter the authors here                   %%
%%                                          %%
%% Ensure \and is entered between all but   %%
%% the last two authors. This will be       %%
%% replaced by a comma in the final article %%
%%                                          %%
%% Ensure there are no trailing spaces at   %% 
%% the ends of the lines                    %%     	
%%                                          %%
%%%%%%%%%%%%%%%%%%%%%%%%%%%%%%%%%%%%%%%%%%%%%%


\author{Franziska Hinkelmann$^{1,2}$%
       \email{Franziska Hinkelmann - fhinkel@vt.edu}%
      \and
         Madison Brandon$^{3,\dagger}$%
         \email{Madison Brandon - mbrando1@utk.edu}
		\and
         Bonny Guang$^{4,\dagger}$%
         \email{Bonny Guang - bonny.guang@gmail.com}
		\and
	     Rustin McNeill$^{5,\dagger}$%
	      \email{Rustin McNeill - rcmcneil@uncg.edu}
		\and
         Grigoriy Blekherman$^1$%
          \email{Grigoriy Blekherman - grrigg@vbi.vt.edu}	
		\and
         Alan Veliz-Cuba$^{1,2}$%
         \email{Alan Veliz-Cuba - alanavc@vt.edu}
       and 
         Reinhard Laubenbacher\correspondingauthor$^{1,2}$%
         \email{Reinhard Laubenbacher\correspondingauthor - reinhard@vbi.vt.edu}%
      }
      

%%%%%%%%%%%%%%%%%%%%%%%%%%%%%%%%%%%%%%%%%%%%%%
%%                                          %%
%% Enter the authors' addresses here        %%
%%                                          %%
%%%%%%%%%%%%%%%%%%%%%%%%%%%%%%%%%%%%%%%%%%%%%%

\address{%
    \iid(1)Virginia Bioinformatics Institute, Blacksburg, VA 24061-0123, USA\\
    \iid(2)Virginia Tech, Blacksburg, VA 24061-0123, USA\\
    \iid(3)University of Tennessee - Knoxville, Knoxville, TN 37996-2513, USA\\
    \iid(4)Harvey Mudd College, Claremont, CA 91711-5901, USA\\
    \iid(5)University of North Carolina - Greensboro, Greensboro, NC 27402-6170, USA\\
    \iid(\dagger)These authors contributed equally
}%

\maketitle

%%%%%%%%%%%%%%%%%%%%%%%%%%%%%%%%%%%%%%%%%%%%%%
%%                                          %%
%% The Abstract begins here                 %%
%%                                          %%
%% The Section headings here are those for  %%
%% a Research article submitted to a        %%
%% BMC-Series journal.                      %%  
%%                                          %%
%% If your article is not of this type,     %%
%% then refer to the Instructions for       %%
%% authors on http://www.biomedcentral.com  %%
%% and change the section headings          %%
%% accordingly.                             %%   
%%                                          %%
%%%%%%%%%%%%%%%%%%%%%%%%%%%%%%%%%%%%%%%%%%%%%%

% Background, Method, Results and Conclusion 
\begin{abstract}
        % Do not use inserted blank lines (ie \\) until main body of text.
        \paragraph*{Background:} Many biological systems are modeled qualitatively with discrete models, such as probabilistic Boolean networks, logical models, Petri nets, and agent-based models, to gain a better understanding of them. The computational complexity to analyze the complete dynamics of these models grows exponentially in the number of variables, which impedes working with complex models. \ul{There exist software tools to analyze discrete models, but they either lack the algorithmic functionality to analyze complex models deterministically or they are inaccessible to many users as they require understanding the underlying algorithm and implementation, do not have a graphical user interface, or are hard to install. Efficient analysis methods that are accessible to modelers and easy to use are needed.}

		\paragraph*{Results:} \ul{We propose a} method for efficiently identifying attractors and \ul{introduce} the web-based tool Analysis of Dynamic Algebraic Models ({\it ADAM}), which provides this and other analysis methods for discrete models. {\it ADAM} converts several discrete model types automatically into polynomial dynamical systems and analyzes their dynamics using tools from computer algebra. Specifically, we propose a method to identify attractors of a discrete model that is equivalent to solving a system of polynomial equations, a long-studied problem in computer algebra. Based on extensive experimentation with both discrete models arising in systems biology and randomly generated networks, we found that the algebraic algorithms presented in this manuscript are fast for systems with the structure maintained by most biological systems, namely sparseness and robustness. For a large set of published complex discrete models, {\it ADAM} identified the attractors in less than one second.
		
		\paragraph*{Conclusions:} Discrete modeling techniques are a useful tool for analyzing complex biological systems and there is a need in the biological community for accessible efficient analysis tools. {\it ADAM} provides analysis methods based on mathematical algorithms as a web-based tool for several different input formats, and it makes analysis of complex models accessible to a larger community, as it is platform independent as a web-service and does not require understanding of the underlying mathematics.
\end{abstract}



\ifthenelse{\boolean{publ}}{\begin{multicols}{2}}{}




%%%%%%%%%%%%%%%%%%%%%%%%%%%%%%%%%%%%%%%%%%%%%%
%%                                          %%
%% The Main Body begins here                %%
%%                                          %%
%% The Section headings here are those for  %%
%% a Research article submitted to a        %%
%% BMC-Series journal.                      %%  
%%                                          %%
%% If your article is not of this type,     %%
%% then refer to the instructions for       %%
%% authors on:                              %%
%% http://www.biomedcentral.com/info/authors%%
%% and change the section headings          %%
%% accordingly.                             %% 
%%                                          %%
%% See the Results and Discussion section   %%
%% for details on how to create sub-sections%%
%%                                          %%
%% use \cite{...} to cite references        %%
%%  \cite{koon} and                         %%
%%  \cite{oreg,khar,zvai,xjon,schn,pond}    %%
%%  \nocite{smith,marg,hunn,advi,koha,mouse}%%
%%                                          %%
%%%%%%%%%%%%%%%%%%%%%%%%%%%%%%%%%%%%%%%%%%%%%%




%%%%%%%%%%%%%%%%%%%%%%%%%%%%%%%%%%%%%%%%%%%%%%
%% Background %%
%%%%%%%%%%%%%%%%%%%%%%%%%%%%%%%%%%%%%%%%%%%%%%
\section*{Background}
Mathematical modeling is a crucial tool in understanding the dynamic behavior
of complex biological systems. Discrete models are now widely used for this
purpose \cite{Sahin2009,Heiser2009,Calzone2010}. Model types include (probabilistic) Boolean networks, logical
networks, Petri nets, cellular automata, and agent-based (individual-based)
models, to name the most commonly found ones \cite{Steggles, heiner-petri,
shmulevich, Chaouiya, cell-automata, springer_book}. 

\hl{Exhaustive enumeration of the state space is common practice to analyze discrete models, but it is limited by computational complexity, as the state spaces grows exponentially in the number of variables. There are several existing software tools that allow users to analyze and simulate discrete networks instead of providing their own implementation of analysis routines} \cite{GINsim,boolnet,Snoopy,DDLab,shmulevich-matlab,Dubrova2010,Ruths2008}. GINsim, (Gene Interaction Network simulation), a computer tool for the modeling and simulation of genetic regulatory networks, can identify steady states for networks with more than $32$ variables, but oscillatory limit cycle analysis is only available through enumeration of the state space, thus limited to small models \cite{GINsim}. \ul{Other tools can deterministically analyze models with 32 variables or less, i.e., $4 \ \times 10^{9}$ states for Boolean systems, or use heuristic search methods, which are not guaranteed to find all attractors, for larger networks. }Each tool has been developed to suit the needs of a particular community and is designed to analyze a different model type. \ul{Algorithms based on satisfiability (SAT) or Binary Decision Diagram (BDD) provide means to analyze complex models, but existing implementations are usually application specific and hardly accessible to non-programmers. To our knowledge, there currently does not exist a software tool that provides algorithms for limit cycle analysis for networks with more than 32 variables and provides a graphical user interface.} A discussion of several state-of-the-art tools is given in {\it Results and Discussion}.

%\comment{ There are several existing software packages for simulation and analysis of discrete models. These include GINsim, BoolNet R, Snoopy, signaling Petri net-based simulator in the PathwayOracle toolkit, DDLab, and BN/PBN Matlab Toolbox \cite{GINsim,boolnet,Snoopy,DDLab,shmulevich-matlab,Dubrova2010,Ruths2008}. Each package has been developed to suit the needs of a particular community, and each package is designed to analyze a different model type. They are discussed in detail in {\it Results and Discussion}.  }



Different types of discrete models, including  (probabilistic) Boolean networks, logical
networks, Petri nets, cellular automata, and agent-based 
models, can be converted into the unifying framework of algebraic models, namely polynomial
dynamical systems \cite{Alan:Bioinf2010, Hinkelmann:2010}. This allows \ul{us} to
apply tools from computational commutative algebra to analyze the models
more efficiently than by simulation and without using heuristic methods. 

\hl{Certainly, many discrete models too complex for analysis by exhaustive enumeration have been published, 
for example on the expression of the segment polarity genes in Drosophila
melanogaster ( $60$ variables) or T-cell regulation ($51$ variables)}
\cite{AO,Pedicini-Tcell}. \hl{The authors have analyzed the dynamics of the models;} \ul{the Drosophila model was analyzed for steady states by manually solving a system of Boolean equations, and an algorithm based on BDD was implemented and used to analyze the model of T-cell regulation. Such application specific analysis is labor-intensive, error-prone, and requires extensive algorithmic and programming expertise.} \hl{We believe that complex discrete models will gain more popularity, if sophisticated analysis methods are easily accessible to modelers.}




%%%%%%%%%%%%%%%%%%%%%%%%%%%%%%%%%%%%%%%%%%%%%%
%%%%%%%%%%%%%%%%%%%%%%%%%%%%%%%%%%%%%%%%%%%%%%
\section*{Results and Discussion}


\hl{In this manuscript, we present the web-based tool {\it ADAM}, Analysis of Dynamic
Algebraic Models} \cite{ADAM}, \hl{a tool to study the dynamics of a wide
range of discrete models. {\it ADAM} provides efficient analysis methods based on mathematical algorithms as a web-based tool for several different input formats, and it makes analysis of complex models accessible to a larger community, as it is platform independent as a web-service and does not require understanding of the underlying mathematics. {\it ADAM} is the successor to DVD, Discrete Visualizer of
Dynamics} \cite{DVD}, \hl{a tool to visualize the temporal evolution of small
polynomial dynamical systems.}

%{\it ADAM} is an analysis tool for discrete models, available as a web-based tool, which hides and encapsulates the mathematical algorithms from the user. Therefore, users who lack understanding of the underlying mathematics or programming expertise can use efficient algorithms to analyze complex models.

%A wide range of software and algorithms exist to analyze discrete models. These tools are either limited by complexity as they rely on simulation as analysis method, or they are inaccessible to biologists not familiar with programming, as they are often times only available as platform dependent command-line tools. Furthermore, implementations require different input formats, which hampers the use of different software tools on the same model. 


 
We propose a novel method to identify attractors of a discrete model. This method relies on the fact that many discrete models can be translated into the algebraic framework of polynomial dynamical systems. Using these polynomials, one can construct a system of polynomial equations, such that its solutions correspond to fixed points or limit cycles. Thus, the problem of identifying attractors becomes equivalent to solving a system of polynomial equations over a finite field. This is a long-studied problem in computer algebra, and can usually be solved efficiently by computing a Gr\"obner basis \cite{IVA}.

This method is not a new mathematical algorithm to solve polynomial dynamical systems, but a novel approach that uses the fact that attractors are the solutions of polynomial systems derived from the model when expressed in the algebraic framework. {\it ADAM} allows users unfamiliar with polynomial dynamical systems or Gr\"obner \ul{bases} to benefit from this efficient algorithm. \ul{We tested the method on several examples and had an average run time of less than one second, comparable to the performance of other software tools; and we were able to identify limit cycles of systems with more than $32$ variables in less than one second, a task which no other software tool available to us was able to perform on a model of that complexity.}


%\ul{With {\it ADAM} it was possible to identified all steady states and small limit cycles
%for these systems
%in less than one second.} In {\it Application}, we show how to use ADAM to carry %out the calculations for the Drosophila model, the model files in {\it ADAM} format can be accessed at \cite{ADAMRepo}. We hope that {\it ADAM} encourages more modelers to build complex methods as it provides them with a simple way of analyzing their models.

%\hl{We used {\it ADAM} on several logical models and on published Boolean models with up to 60 variables} \cite{GINsimRepo,AO,Pedicini-Tcell}. \hl{These models are too large for a straight-forward analysis by exhaustive
%enumeration of the state space, and the corresponding publications contain lengthy explanations and supplementary material that outline the calculations and algorithms used to identify the attractors. {\it ADAM} greatly simplifies the analysis; it
%identified the steady states of these models in less than one second.}


\hl{In
addition to providing access to mathematical theory for efficient analysis, algebraic models are a unifying framework and systematic approach for several model types. \ul{This} allows for an
effective comparison of heterogeneous models, such as a Boolean network model
and an agent-based model. For community integration to the biological
sciences, {\it ADAM} contains a model repository of previously published models
available in {\it ADAM} specific format} \cite{ADAMRepo}. \hl{This allows new users to familiarize themselves quickly with {\it ADAM} and to validate and experiment with existing
models. In the following section, we discuss general features of {\it ADAM} briefly and
explain new features in more detail.}

%%%%%%%%%%%%%%%%%%%%%%%%%%%%%%%%%%%%%%%%%%%%%%
\subsection*{General Features of ADAM}
\ul{{\it ADAM} is a tool for analyzing different types of discrete models. It} automatically converts discrete models into polynomial dynamical systems,
that is time and state discrete dynamical systems described by polynomials
over a finite field (see Appendix \ref{sec:pds} for definition and example). The dynamics of the models are then analyzed by using various computational algebra
techniques. Even for large systems, {\it ADAM} computes key dynamic features, such
as steady states, in a matter of seconds. {\it ADAM} is available online and free of
charge. It is platform independent and does not require the installation of
software or a computer algebra system.

%% different input types
\ul{{\it ADAM}} translates the following {\bf inputs} into (probabilistic) polynomial dynamical systems and can then analyze \ul{them}.
\begin{itemize}
  \item Logical models generated with GINsim \cite{GINsim}
  %\comment{\item Petri nets generated with Snoopy \cite{Snoopy}}
  \item polynomial dynamical systems
  \item Boolean networks
  \item probabilistic polynomial dynamical systems, probabilistic Boolean networks (PBN) \cite{shmulevich}.
\end{itemize}
\ul{{\it ADAM} also translates Petri nets generated with Snoopy and we }plan to implement analysis methods for Petri nets in future versions.

{\it ADAM}’s main application is the analysis of the dynamic features of a model, which includes the identification of stable attractors. These are either steady states, i.e.,
time-invariant states, or limit cycles, i.e., time-invariant sets of states. {\it ADAM} is capable of identifying all steady states and
limit cycles of length up to a user-specified length $m$. The process of
finding long limit cycles is quite slow for large models, however, in
biological models limit cycles are likely to be short, so that $m$ can be
chosen to be small in general, i.e., less than ten.

The temporal evolution of the model can be visualized by the {\it phase
space}, a graph of all possible states and their transitions, also called {\it state space} or {\it state transition graph}. For small enough
models, i.e., less than eleven variables, {\it ADAM} generates a graph of the complete phase space; for larger models, {\it ADAM} uses algebraic algorithms to determine dynamic properties. Independent of
network size, {\it ADAM} generates a {\it wiring diagram}. The wiring diagram, also
known as {\it dependency graph}, shows the static relationship between the
variables. All edges in {\it ADAM}’s wiring diagrams are functional edges, that is
there exists at least one state such that a change in the input variable causes a
change in the output variable (see Appendix \ref{sec:func} for more details).
This means that {\it ADAM} determines all non-functional edges, which is oftentimes
of interest.

With {\it ADAM}, one can also study the temporal evolution of user-specified initial states. The trajectory of a state describes the state's evolution, and it can be computed by repeatedly applying the transition function until an attractor is reached.  

All of these features can be computed assuming synchronous updates or sequential updates according to an update-schedule specified by the user. Note that the steady
states are the same independent of the update schedule. This is due to the
fact that updating any variable at a steady state does not change its value.
It is irrelevant for a steady state analysis whether updates are considered to happen sequentially or simultaneously.

For probabilistic networks, i.e., models in which each variable has several choices of local update rules, {\it ADAM} can generate a graph of all possible updates. This means that states in the phase space can have out-degree greater than one, since different transitions are possible. {\it ADAM} can find all true steady states, in the context of probabilistic networks, meaning all states that are time-invariant independent of the choice of update function. For further information of probabilistic networks, see \cite{shmulevich}.

For Boolean networks, {\it ADAM} calculates all functional circuits (see Appendix \ref{sec:func}). Positive functional circuits are a necessary condition for multi-stationarity. For a certain class of Boolean networks, namely conjunctive/disjunctive networks, {\it ADAM} computes a complete description of the phase space as described in \cite{conjunctive}. For further details on conjunctive networks, see Appendix \ref{sec:conj}.

In summary, {\it ADAM} can generate the following {\bf outputs}.
\begin{itemize}
\item wiring diagram
\item phase space for small models
\item steady states (for deterministic and probabilistic systems)
\item limit cycles of specified length $m$
\item trajectories originating from a given initial state until a stable
attractor is found
\item dynamics for synchronous or sequential updates
\item functional circuits for Boolean networks
\item a complete description of the phase space for conjunctive/disjunctive
networks.
\end{itemize}

%%%%%%%%%%%%%%%%%%%%%%%%%%%%%%%%%%%%%%%%%%%%%%


%*************************************
\subsection*{Application} \label{benchmarks}
% Example application
We show how to use {\it ADAM} on a well-understood model of the expression
pattern of the segment polarity genes in Drosophila melanogaster. Albert and Othmer developed a model for embryonic pattern formation in the fruit fly Drosophila melanogaster \cite{AO}. Their Boolean
model consists of 60 variables, resulting in a state
space with more than $10^{18}$ states. They analyze the model for steady states by manually solving a system of Boolean equations. They also analyze the temporal evolution of a specific initial state corresponding to the wild type expression pattern by repeatedly applying the Boolean update rules until a steady state is found. The update schedule of the model is synchronous  with the exception of activation of SMO and the binding of PTC to HH (activation of PH), which are assumed to happen instantaneously. This can be accounted for by substituting the equations for SMO and PH into the update rules for other genes and proteins, rather than using SMO and PH themselves.

To analyze the model, we first rename the variables in the Boolean rules given in \cite{AO} such as $\text{wg}_i$ or $\text{SLP}_i$ to $x_1 \ldots x_{60}$, to standardize their format. \ul{The variables $x_i$ and their corresponding genes are listed in table 1}. Then we use {\it ADAM}: the model type is {\it Polynomial Dynamical Systems}, the number of states in a Boolean model is $2$, representing present or absent. One can choose {\it Boolean}, and enter the Boolean rules in the text-area or upload a text file with the Boolean rules. Alternatively, one can first convert the Boolean rules to polynomials over $\mathbb F_2$, and enter the polynomials with the choice {\it Polynomial}.
The file with the polynomial equations for the model can be accessed at \cite{ADAMRepo}. 

The rules in the model file are specified in {\it Polynomial} form. Once the polynomials are uploaded, we need to set the {\it Analysis} type. The model with $60$ variables is too complex for exhaustive enumeration, and we choose {\it Algorithm}. This means that instead of exhaustive enumeration of the state space, analysis of the dynamics is done via computer algebra by solving systems of equations.
In {\it Options}, we set {\it Limit cycle length} to one because we are interested in the steady states, i.e., time-invariant states.
We chose {\it synchronous} as updating scheme. Once these choices have been made,
we obtain the steady states by clicking {\it Analyze}.
{\it ADAM} returns a link to the {\it wiring diagram} (or {\it dependency graph}), which captures the static relations between the different variables. In addition, {\it ADAM} returns the number of steady states and the steady states themselves, see figure 1. These steady states are identical to those found in \cite{AO}, half of which have been observed experimentally.

Each row in the table in figure 1 corresponds to a stable attractor. Attractors are written as binary strings, where $0$ represents non-expression of a gene (or low concentration of a protein), and $1$ expression (or high concentration), e.g., the binary string 
\begin{align}\label{equ:ss}
	(&0 0 0 1 1 1 1 0 0 0 1 0 0 0 0
	0 0 0 0 0 0 0 1 1 1 1 1 1 1 0\notag\\
	&1 0 0 0 0 0 0 0 1 0 0 1 1 0 1
	1 1 1 0 0 0 0 1 1 1 1 1 1 1 0)
\end{align}
corresponds to the genes (and proteins) being expressed (or present in high concentration) in four cells from anterior to posterior compartments (compartment $1$ to $4$). \ul{The string can be translated back to a list of genes that are expressed in this stable attractor, see table 2}.
This is the steady state obtained in \cite{AO} when starting the system with an initial state representing the experimental observations of stage 8 embryos. 

{\it ADAM} can also generate trajectories for a given initial state. For example, we
can choose the initial state that was used in \cite{AO} representing stage 8 embryos. Again, we
enter {\it Polynomial Dynamical Systems} with $2$ as the number of states and upload the polynomials
describing the model. Instead of {\it Algorithms}, we now choose {\it
Simulation}. Since we are not interested in the number of steady states or the
complete phase space, but in a single trajectory originating from a specific
initial state, we choose {\it One trajectory starting at an initial state} as the simulation option. \ul{We enter the state corresponding to the initial state shown in table} 3 as a binary string: 
\begin{align}\label{equ:ini}
	(&0 0 0 1 0 1 0 0 0 0 0 0 0 0 0
	0 0 0 0 0 0 0 1 0 0 0 1 0 0 0\notag\\
	&1 0 0 0 0 0 0 1 0 0 0 1 0 0 0
	1 1 0 0 0 0 0 1 0 0 0 1 0 0 0).
\end{align}
By clicking {\it Analyze}, we obtain the temporal evolution of this particular
state until it reaches a steady state, see figure
2. As predicted in \cite{AO}, the
steady state is the state corresponding to the state shown in table 2.


To summarize, {\it ADAM} correctly identified the steady states
in less than one second. All steady states have been determined previously in \cite{AO} by labor-intensive manual investigation of the system. 

Furthermore, we used {\it ADAM} to verify that there are no limit
cycles of length two or three. The model has not been analyzed previously for
limit cycles. The absence of two- and three-cycles strengthens confidence in
the model, since oscillatory behavior has not been observed experimentally. Computations for limit cycles of length greater than three have not been conducted, as composing the system several times with itself is computationally complex.
The model file in {\it ADAM} format can be accessed at \cite{ADAMRepo}.

%%%%%%%%%%%%%%%%%%%%%%%%%%%%%%%%%%%%%%%%%%%%%%

\subsection*{Benchmark Calculations}
We analyzed logical models
available in the GINsim model repository \cite{GINsimRepo} as of August 2010. The
repository consists of models in GINsim XML format previously published in
peer-reviewed journals. We converted all but two models into polynomial
dynamics systems. For these 26 models we computed the steady states. All
calculations finished in less than 1,5 seconds, see
figure 3.

In addition to the published models in \cite{GINsimRepo}, we analyzed
randomly generated networks
that have the same structure that we
expect from biological systems, namely sparse, i.e., while the number of nodes in a biological network may be quite large, each node is affected only by a small number of other nodes, and robust, i.e., small number of attractors. We tested a total of 50 networks with
50-150 nodes ($10^{15} - 10^{45}$ states) and an average of average in-degrees of $1.6848$. The
steady state calculations took less than half a second for each network on
a 2.7 GHz computer.



%%%%%%%%%%%%%%%%%%%%%%%%%%%%%%
\subsection*{Comparison to Other Systems}
%proper evaluation and comparison to other systems to warrant further consideration. A systematic assessment of quality of the models generated needs to be provided, and the paper needs to convincingly show advance over state-of-the-art systems.
In this section, we describe the functionality of several state-of-the-art software tools for the analysis of discrete models of biological systems. They are all capable of identifying steady states and limit cycles by exhaustive enumeration of the state space for small models ( less than $32$ variables) \cite{GINsim,boolnet,DDLab,shmulevich-matlab,Dubrova2010,Ruths2008}. For larger models, GINsim is capable of analyzing models for steady states, and several tools provide heuristic analysis methods. None of them identifies limit cycles deterministically for models with more than $32$ variables. \ul{It is important to stress that ADAM provides a web-interface and does not require local installation as all the other tools do, which makes them less accessible to users. Table 4 summarizes the features of the different software tools, which we will now explain in detail. }


{\it GINsim} (Gene Interaction Network simulation) is a package designed for the
analysis of gene regulatory networks \cite{GINsim}. As input, it accepts
logical models. Logical models are an extension to Boolean models; they
consist of similar switch-like rules, but allow for a finer discretization with more
than two states per variable, e.g., \ul{{\it low}, {\it medium}, and {\it high}}. Logical models can be updated synchronously or asynchronously. For the latter, the temporal
evolution of a logical model is non-deterministic because the variables are updated randomly in an asynchronous
fashion. In either case, updates of every variable are {\it \ul{continuous}}, meaning that no variable changes its value by more than one unit in one time-step, see section {\it Remarks about Logical Models} for a detailed discussion. 

GINsim
provides algorithms that use binary decision diagrams (BDD) for the
determination of steady states\cite{Chaouiya}. 
Analysis of limit cycles \ul{is executed by} simulating every trajectory, i.e., generating the complete state space, called state transition graph in GINsim,
and therefore limited by network size.
We tested GINsim on logical models with up to $72$ variables; determining the
steady states took less than one second. More complex logical networks were not available to us.

%Networks are entered manually into GINsim, they cannot be imported from any other format. Furthermore, models are specified by entering their parameters, i.e., entering all values that result in a non-zero target value. Especially for large models, this can be a time consuming process. 


{\it BoolNet R package} provides methods for inference and analysis of synchronous,
asynchronous, and probabilistic Boolean networks \cite{boolnet}. It is a
package for the free statistics software {\it R}, and it is run via the {\it R} command-line. It is helpful, if the user is already familiar with {\it R}. 
Steady state
analysis is implemented as exhaustive search of the state space, heuristic
search, random walk, or Markov Chain analysis \cite{shmulevich}. Non-heuristic analysis is limited to networks with 29 variables.
For
larger networks, steady states can be inferred heuristically, which does not guarantee that all steady states are identified. 


%\comment{{\it Snoopy} is a unifying Petri net framework, containing a family of Petri net modeling tools and algorithms \cite{Snoopy}. Snoopy provides built-in simulation
%and animation. Analysis of Petri nets can be performed, e.g., with the
%tool {\it Charlie} \cite{Charlie}. Charlie identifies structural properties and
%has algorithms for invariant based or reachability graph based analysis. Reachability graph-based analysis for Petri nets usually depends on a given initial state and does not provide a complete picture of possible dynamics for other initial markings. In addition to marking-dependent analysis, Charlie uses algorithms based on linear algebra to predict the dynamic properties independent of markings, such as T and P-invariants. {\it ADAM} converts a Petri net to a collection of polynomial dynamical systems, one system for each transition. Analysis of such a non-deterministic system is currently not implemented in {\it ADAM}, and we therefore do not list any run-time results for Petri nets.} 

{\it DDLab} is an interactive graphics software for discrete models, including
cellular automata, Boolean and multi-valued networks \cite{DDLab}. As it is
mainly a visualization tool, analysis is based on exhaustive enumeration of
the state space, and model size is limited to 31 variables.  

%\comment{{\it GenYsis-P Toolbox} is a command-line tool currently available only for Linux to analyze (probabilistic) gene regulatory networks \cite{Garg2008}. Algorithms use (reduced order) binary decision diagrams. As analysis methods are not based on exhaustive enumeration, GenYsis-P can analyze large networks. Unfortunately, we did not have access to a platform that supports the distributed binaries, and source code was not available.}


{\it BN/PBN Toolbox} is a toolbox written in {\it Matlab} \cite{shmulevich-matlab}.
It uses the state transition matrix to compute attractors. Statistics for
networks with more than 27 variables cannot be computed (``Maximum variable
size allowed by the program is exceeded"). In addition to analyzing
deterministic Boolean networks, the toolbox can analyze probabilistic Boolean
networks and calculate statistics such as numbers and sizes of attractors,
basins, transient lengths, Derrida curves, percolation on 2-D lattices, and
influence matrices. 




%*************************************
\subsubsection*{Remarks about Logical Models} 
\ul{In this manuscript, we distinguish between three different update types: synchronous, sequential according to an update schedule, and asynchronous.}
{\it ADAM} allows for synchronous or sequential updates according to a given update schedule. In models with synchronous updates, all variables are updated simultaneously at every time step. In models with sequential updates according to an update schedule, all variables are updated at every time-step in the order given by the schedule. \ul{Both these models are deterministic. For each model with sequential updates and a given update schedule, there exists a synchronous model with identical state space. For example, the model $(f_1(x_1, x_2), f_2(x_1, x_2))$ with sequential update schedule $(1,2)$, i.e., first update $x_1$ then $x_2$, has the same state space as the synchronous model $(f_1(x_1, x_2), f_2(f_1(x_1, x_2), x_2))$.} In models with asynchronous updates, as is common for logical models, one variable is updated at random at every time step, which results in a non-deterministic model. \ul{Models with 
sequential updates according to an update schedule produce different dynamics than models with asynchronous updates, i.e., logical models.}

In GINsim, all models are {\it \ul{continuous}} in the sense that at each time-step, each variable increases or decreases by at most one unit. Though logical models are discrete, there are no jumps skipping intermediate states. For example, in a model with three states, low, medium, and high, no variable can drop from high to low in a single update step. This interpretation is different from the common meaning of continuous, which usually refers to models of ordinary or partial differential equations. The parameters entered in GINsim specify the target value towards which the variable changes, i.e., the value increases by one, decreases by one, or remains constant if the target value is larger, smaller, or equal than the initial value, respectively. The phase space generated with {\it ADAM} might differ from the state transition graph generated in GINsim. To obtain the exact same phase space, every variable in the logical model must contain an explicit self-loop, and all parameters must be entered such that the target value differs by at most one from the value of the variable to be updated. Any logical model can be specified in this way without changing its state transition graph. Boolean models are always continuous.   

In multi-valued logical models, variables can have different maximum values. In an algebraic model, all variables are defined over the same algebraic field, i.e., have the same maximum value. When a multi-valued logical model is translated into an algebraic model, extraneous states might be introduced such that all variables are defined over the same field. An example of such an extension is given in table 5, the extra states are the states in the last row, which are given the same values as the states above to extend the model in a meaningful way. The extra states should be ignored when analyzing the dynamics. For more details, see \cite{Alan:Bioinf2010}.  




%**************
%\comment{\subsubsection*{Remarks about Petri Nets}
%In the Petri net community, {\it state space} usually refers to all states (markings) reachable for a given initial state. In this manuscript, {\it state space} refers to the set of all possible states, independent of an initial state.

%Translating a bounded Petri net to an algebraic model results in a set of polynomial dynamical systems, where every transition corresponds to one system. In a Petri net, different firing sequences can lead to different markings; a firing sequence relates to the order in which the different systems are iterated. The {\it update schedule} is not related to the firing sequence. As firing does not consume any time, polynomial systems describing a Petri net always use synchronous updates \cite{Heiner:2008}. 

%The term {\it functional edge} is not related to the concept of {\it liveness of a transition}. The liveness of a transition depends on an initial marking. An edge, connecting two places (source and target), is functional, if there exists a marking, such that changing only the marking of the source place, changes the marking of the target place (see Appendix \ref{sec:func} for more details on functional edges). }

%%%%%%%%%%%%%%%%%%%%%%%%%%%%%%%%%%%%%%%%%%%%%%%%%%%%%%%%%%%%%%%%%%
\subsection*{Architecture}
{\it ADAM} is available as an web-based tool, that does not require any software installation. {\it ADAM}'s user interface is implemented in HTML. We use JavaScript to generate a dynamic website that adapts as the user makes various choices. This simplifies the process of entering a model. For example, after defining the model type, i.e., Polynomial Dynamical System, Probabilistic Network, Petri net, and Logical Model the next line changes to the number of states, $k$-bound, or nothing, appropriately. Input can be entered directly into the text-area, or uploaded as a text document.

All mathematical algorithms are programmed in Macaulay2 \cite{M2}. Macaulay2 is a powerful computer algebra system. The routines for which fast execution is crucial are implemented in C/C++ as part of the Macaulay2 core. Logical Models and Petri nets in XML format are parsed using Ruby's XmlSimple library. The interplay between HTML and Macaulay2 is also programmed in Ruby.

Output graphs are generated with Graphviz's {\it dot} command. When {\it Simulation} is chosen as analysis method, Graphviz's {\it ccomps - connected components filter for graphs} is used to count the connected components. A Perl script directs the execution of the Graphviz commands.
%%%%%%%%%%%%%%%%%%%%%%%%%%%%%%%%%%%%%%%%%%%%%%%%%%%%%%%%%%%%%%%%%%
\subsection*{Model Repository}
A model repository is part of the {\it ADAM} website \cite{ADAMRepo}. The repository consists of a collection of several previously published models in {\it ADAM} format. The models are extracted from publications, and rewritten in {\it ADAM} specific format, i.e., all variables are renamed to $x_i$ and the update rules from the original publication are reformulated as Boolean rules or polynomials. The central repository with models in a unified framework allows for quick verification and experimentation with published models. By changing parameters or initial states, users can gain a better understanding of the models.

New users can also use the repository to quickly familiarize themselves with the main functionalities of {\it ADAM}. In addition to the model itself, the database entries contain a short summary of the biological system and relevant graphs, together with an analysis of dynamic features determined by {\it ADAM} and their biological explanation. The repository is work in progress by researchers from several institutions generating more entries for the repository. We invite all interested researchers to submit their models.

Because of their intuitive nature, discrete models are an excellent introduction to mathematical modeling for students of the life sciences. {\it ADAM}'s model repository is a great starting point to familiarize students with the abstraction of discrete models such as Boolean networks.
%%%%%%%%%%%%%%%%%%%%%%%%%%%%%%%%%%%%%%%%%%%%%%%%%%%%%%%%%%%%%%%%%%
 
%%%%%%%%%%%%%%%%%%%%%%%%%%%%
%% Results and Discussion %%
%%
%\section*{Results and Discussion}
% \subsection*{Results sub-heading}
%    \subsubsection*{This is a sub-sub-heading}
%      Sub-sub-sub-headings are made with the \textsl{\\subsubsection} command. \pb
%      pb at end of lines ensures correct paragraph spacing.\pb%
%	  Text for this sub-sub-section \ldots
%    \subsubsection*{Another sub-sub-sub-heading}
%      Text for this sub-sub-section \ldots%
%
%  \subsection*{Another results sub-heading}
%    Text for this sub-section \ldots
%
%  \subsection*{Yet another results sub-heading}
%    Text for this sub-section.  More results \ldots


    

%%%%%%%%%%%%%%%%%%%%%%
\section*{Conclusions}
Discrete modeling techniques are a useful tool for analyzing complex biological systems and there is a need in the biological community for easily accessible analysis software. {\it ADAM} provides efficient methods as a web-based tool and will allow a larger community to use complex modeling techniques, as it is platform independent and does not require the user to understand the underlying mathematics.
Upon translating discrete models, such as logical networks,
Petri nets, or agent-based models into algebraic models, rich mathematical
theory becomes available for model analysis.

After extensive experimentation with both discrete models arising in systems biology and randomly generated networks, we found that the algebraic algorithms presented in this manuscript
are fast for sparse systems with few attractors, a structure maintained by most biological
systems. All algorithms have been included in the software package {\it ADAM}\cite{ADAM},
which is user-friendly and available as a free web-based tool.
{\it ADAM} is highly suitable to be used in a classroom as a first
introduction to discrete models because students can use it without going through an installation process.

{\it ADAM} provides methods to analyze the key dynamic features, such as steady states and limit cycles, for large-scale (probabilistic) Boolean networks and logical models. {\it ADAM} unifies different modeling types by providing analysis methods for all of them and thus can be used by a larger community.

We hope to expand {\it ADAM} to a more comprehensive Discrete Toolkit which incorporates new analysis methods, better visualization, and automatic conversion for more model types.
We also hope to analyze controlled algebraic models and expand theory to stochastic systems.




  
%%%%%%%%%%%%%%%%%%
\section*{Methods}
Logical models, Petri nets, and Boolean networks are converted automatically
into the corresponding polynomial dynamical system as described in
\cite{Alan:Bioinf2010}, so that algorithms from computational
algebra can be used to analyze the dynamics. In polynomial dynamical systems over a finite field, states of a variable are assigned to values in the field, and the update (or transition) rule for each variable is given as a polynomial rather than a Boolean or logical expression. For more details, see appendix \ref{sec:pds}. Using these polynomials, one can construct systems of polynomial equations, such that their solutions correspond to fixed points or limit cycles. Thus, the problem of identifying attractors becomes equivalent to solving a system of polynomial equations over a finite field. This is a long-studied problem in computer algebra, and can usually be solved efficiently by computing a Gr\"obner basis.

Gr\"obner basis calculation is for polynomial systems what
Gauss-Jordan elimination is for linear systems: a structured way to transform
the original system to triangular shape without changing its solution space.
The triangular shape of the resulting systems allows for stepwise retrieval of the solutions of the system. For a more in depth discussion of Gr\"obner bases, see for example \cite{IVA}.

In the worst case, computing Gr\"obner bases for a set of polynomials has
complexity doubly exponential in the number of solutions to the system.
However, in practice, Gr\"{o}bner bases are computable in a reasonable time. It has been suggested, that in robust gene
regulatory networks genes are regulated by only a handful of regulators
\cite{Leclerc:2008}. Thus, the polynomial dynamical systems representing such biological networks are
sparse, i.e., each function depends only on a small subset of the model variables. From our experience, a Gr\"obner basis calculation 
for sparse systems with few attractors, a structure common for
biological systems, is actually quite fast. 

%**********************************************
%
%the reader is likely to encounter this term for the first time, so we need a reference or
%explanation here. - we moved up the description from below and cite iva
%
%
%*******************************************


%*******************************************
\appendix
\section{Mathematical Background}
\subsection{Polynomial Dynamical Systems}\label{sec:pds}
To be self-contained, we briefly explain polynomial dynamical systems and their key features.

%\subsubsection*{Polynomial Dynamical System (PDS)}\label{pds}
A {\bf polynomial dynamical system} (PDS) \cite{JLSS} over a finite field $k$ is a function
$$f = (f_1, \ldots, f_n) : k^n \rightarrow k^n,$$
with coordinate functions $f_i \in k[x_1, \ldots , x_n]$. Iteration of $f$ results
in a time-discrete dynamical system. A PDS can be used to describe the dynamic behavior of a biological system: every variable $x_i$ corresponds to a biological substrate, for example a protein or gene, and the polynomials $f_i$ describe the evolution of $x_i$ depending on the previous state of the variables $x_1, \ldots, x_n$.


PDS have several dynamic features of biological
relevance. These include the number of components, component sizes, steady states, limit cycles, and limit cycle lengths.
\begin{example}
Let $k= \mathbb F_2$ and $f = (f_1, f_2, f_3) : \mathbb F_2^3 \rightarrow
\mathbb F_2^3$ with
\begin{align*}
f_1 &= x_1x_2x_3+x_1x_2+x_2x_3+x_2 \\
f_2 &= x_1x_2x_3+x_1x_2+x_1x_3+x_1+x_2 \\
f_3 &= x_1x_2x_3+x_1x_3+x_2x_3+x_1+x_2.
\end{align*}
The wiring diagram of $f$, which shows the static interaction of the three
variables, is
depicted in figure 4 (left) along with its phase space in figure
4 (right).
The phase space shows the temporal evolution of the system. Each state is
represented as a vector of the values of the three variables $(x_1, x_2,
x_3)$.
The PDS described by $f$ has
two stable attractors: a steady state, $(000)$, and a limit cycle of length
three, consisting of the states $(010)$, $(111)$, and $(011)$.
\end{example}


%*******************************************
%\subsubsection*{Probabilistic Polynomial Dynamical System}
A {\bf probabilistic PDS} over a finite field $k$ is a collection of functions
$$f = (\{f_{1,1}, \ldots, f_{1, r_1}\}, \ldots, \{f_{n, 1}, \ldots, f_{n, r_n}
\}) : k^n \rightarrow k^n,$$
together with a probability distribution for every coordinate that assigns the
probability that a specific function is chosen to update that coordinate.
The coordinate functions $f_{i,j}$ are elements in $k[x_1, \ldots , x_n]$.
Probabilistic PDS, specifically Boolean probabilistic networks (PBN), have been studied
extensively in \cite{shmulevich}.
{\it ADAM} analyzes probabilistic PDS. It can simulate the
complete phase space for small enough models, by generating every possible
transition and labeling the edge with its probability according to the
distribution. If no distribution is given, {\it ADAM} assumes a uniform distribution
on all functions. For large networks, {\it ADAM}'s {\it Algorithm} choice computes
steady states of probabilistic networks.
%%%%%%%%%%%
\subsection{Functional Edges} \label{sec:func}
An edge in the wiring diagram from $x_i$ to $x_j$ is considered
functional, if there exists a state $\hat x = (\hat x_1,  \ldots, \hat x_n)$ such
that $f_j( \hat x_1,  \ldots, a, \ldots \hat x_n) \neq f_j(\hat x_1, \ldots, b, \ldots
\hat x_n)$, where $a$ and $b$ are values for $x_i$, in other words, if there
is at least one state, such that changing only $x_i$ but keeping all other
values fixed, changes the next state of $x_j$.
In {\it ADAM}, all edges in the wiring diagram are functional.
For Boolean networks, {\it ADAM} identifies all functional elementary circuits. An elementary circuit is a finite closed paths in the wiring diagram where all the nodes are distinct. Functional Circuits are a necessary condition for multi-stationarity and limit cycles. For a further discussion of
functional circuits, see \cite{Chaouiya}. For multivalued networks, circuit analysis has not yet been implemented.
%%%%%%%%%%%%%%%%%%%%%%%%%%%%%%%%%%%%%%%%%%%%%%%%%%%%%%%%%%%%%%%%%%
\section{Algorithms}
\subsection{Analysis of stable attractors}
Every attractor in a PDS is either a
steady state or a limit cycle. For small models, {\it ADAM} determines the complete
phase space by enumeration, for large models, {\it ADAM} computes steady states and
limit cycles of a given length.
A state is a steady state, if it transitions to itself after one update of the
system. A state is part of a limit cycle of length $m$, if,
after $m$ updates, it results in itself. Any steady state of a PDS satisfies
the equation $f(x) = x$, as no coordinate of $x$ is changing as it is updated.
Similarly, states of a
limit cycle of length $m$ satisfy the equation $f^m(x) = x$. {\it ADAM} computes all
steady states by solving the system $f_i(x) - x_i = 0$ for $i \in \{1, \ldots,
n\}$ simultaneously. To efficiently solve the resulting systems of polynomial
equations, we first compute the Gr\"obner
basis in lexicographic order for the ideal generated by the equations.
By the elimination and extension theorem, choosing a lexicographic order
allows to easily obtain the solutions \cite{IVA}.
We use the Gr\"obner basis algorithms distributed with Macaulay2, a
computer algebra system, and found that for quotient rings over a finite field
the implementation `Sugarless' is more efficient than the default algorithm
with `Sugar' \cite{M2,Sugar:1991}.
For limit cycles of length $m$, the solutions of $f^m(x)=x$ are found and then
grouped into cycles, by applying $f$ to each of the solutions.

\begin{example}
	Fixed points of the system shown in the example in \ref{sec:pds} are solutions in $\mathbb F_2^3$ of the system $f(x)=x$:
	\begin{align*}
	 x_1x_2x_3+x_1x_2+x_2x_3+x_2  &= x_1\\
	 x_1x_2x_3+x_1x_2+x_1x_3+x_1+x_2 &= x_2 \\
	 x_1x_2x_3+x_1x_3+x_2x_3+x_1+x_2 &= x_3.
	\end{align*}
	The only solution to this systems is the point $(x_1, x_2,x_3 ) = (0,0,0)$. This is in accordance with the state space depicted in figure 4: $(0,0,0)$ is the only fixed point. To investigate limit cycles of length two, one has to look at the system $f^2(x) = x$,
	\begin{align*}
	g_1(x) &= f_1(f_1(x), f_2(x), f_3(x) ) \\
		&= x_1*x_2+x_2*x_3 = x_1 \\
	g_2(x) &= f_2(f_1(x), f_2(x), f_3(x) ) \\
		&= x_1*x_2*x_3+x_1*x_2+x_1*x_3+x_1+x_2 = x_2\\
	g_3(x) &= f_3(f_1(x), f_2(x), f_3(x) ) \\
		&= x_1*x_2*x_3+x_2 = x_3.
	\end{align*}
	Again, $(0,0,0)$ is the only solution, which means that there are no limit cycles of length two. 

	Investigating $f^3(x) = x$,
	\begin{align*}
	f_1(g_1(x), g_2(x), g_3(x) ) &= x_1 \\
	f_2(g_1(x), g_2(x), g_3(x) ) &= x_2\\
	f_3(g_1(x), g_2(x), g_3(x) ) &= x_3, 
	\end{align*}
	results in the solutions $(0, 0, 0), (0, 1, 0), (0, 1, 1), (1, 1, 1)$. $(0,0,0)$ is a fixed point, and $(0, 1, 0), (0, 1, 1), (1, 1, 1)$ are elements of a limit cycle of length three. For all $m > 3$, $f^m(x)=x$ has no solutions, that means the system $f$ has exactly two attractors, a fixed point a a limit cycle of length three. 
\end{example}
%%%%%%%%%%%
\subsection{Conjunctive/Disjunctive Networks} \label{sec:conj}
Some classes of networks have a certain structure that can be
exploited to achieve faster calculations. Jarrah et al.
show that for conjunctive (disjunctive) networks key dynamic features can be found with
almost no computational effort \cite{conjunctive}. Conjunctive (respectively disjunctive) networks consist of
functions using only the AND (respectively OR) operator.
{\it ADAM} comes with an implementation of this algorithm to analyze
dynamics in the case of conjunctive (disjunctive) networks. Currently,
this option is only implemented for networks with strongly connected dependency graphs.


    
%%%%%%%%%%%%%%%%%%%%%%%%%%%%%%%%
%\section*{Authors contributions}
 %   Text for this section \ldots

%%%%%%%%%%%%%%%%%%%%%%%%%%%
\section*{Authors' contributions}
FH led the algorithm and software development group. BG, MB, and RM implemented the user interface and attractor analysis, executed benchmarking calculations, and drafted initial manuscript under FH's leadership. AV implemented the translation for logical algorithms to PDS used by {\it ADAM}. GB participated in the software design effort and
algorithm development. RL  conceived of the study, provided overall leadership of the project, and secured
funding for it. He also contributed to the writing and editing of the manuscript. 
All authors read and approved the final manuscript.

%%%%%%%%%%%%%%%%%%%%%%%%%%%
\section*{Acknowledgements}
  \ifthenelse{\boolean{publ}}{\small}{}
 Dimitrova, Clemson University; J. Adeyeye, Winston-Salem State University; B. Stigler, Southern Methodist University; R. Isokpehi, Jackson State University are currently expanding {\it ADAM}’s Model Repository.
Funding for this work was provided through U.S. Army Research Office Grant Nr. W911NF-09-1-0538,
National Science Foundation Grant Nr. CMMI-0908201, and National Science Foundation Grant Nr. 0755322.


 
%%%%%%%%%%%%%%%%%%%%%%%%%%%%%%%%%%%%%%%%%%%%%%%%%%%%%%%%%%%%%
%%                  The Bibliography                       %%
%%                                                         %%              
%%  Bmc_article.bst  will be used to                       %%
%%  create a .BBL file for submission, which includes      %%
%%  XML structured for BMC.                                %%
%%                                                         %%
%%                                                         %%
%%  Note that the displayed Bibliography will not          %% 
%%  necessarily be rendered by Latex exactly as specified  %%
%%  in the online Instructions for Authors.                %% 
%%                                                         %%
%%%%%%%%%%%%%%%%%%%%%%%%%%%%%%%%%%%%%%%%%%%%%%%%%%%%%%%%%%%%%


{\ifthenelse{\boolean{publ}}{\footnotesize}{\small}
 \bibliographystyle{bmc_article}  % Style BST file
  \bibliography{ADAM} }     % Bibliography file (usually '*.bib' ) 

%%%%%%%%%%%

\ifthenelse{\boolean{publ}}{\end{multicols}}{}

%%%%%%%%%%%%%%%%%%%%%%%%%%%%%%%%%%%
%%                               %%
%% Figures                       %%
%%                               %%
%% NB: this is for captions and  %%
%% Titles. All graphics must be  %%
%% submitted separately and NOT  %%
%% included in the Tex document  %%
%%                               %%
%%%%%%%%%%%%%%%%%%%%%%%%%%%%%%%%%%%

%%
%% Do not use \listoffigures as most will included as separate files

\section*{Figures}
  \subsection*{Figure 1 - {\it ADAM}: Analysis of steady states of Drosophila model. Each row in the table corresponds to a stable attractor. Attractors are written as binary strings, where 0 represents non-expression of a gene (or low concentration of a protein), and 1 expression (or high concentration)}
      Steady states of Drosophila Melanogaster as found with {\it ADAM}.

	\begin{figure}[htb]
		\centering
		%\includegraphics{DroAlgOutput.jpg}
		\includegraphics[width=0.95\textwidth]{DroAlgOutput.jpg}

		\caption{{\it ADAM}: Analysis of steady states of Drosophila model. Each row in the table corresponds to a stable attractor. Attractors are written as binary strings, where $0$ represents non-expression of a gene (or low concentration of a protein), and $1$ expression (or high concentration)}
		\label{fig:alg}
	\end{figure}



  \subsection*{Figure 2 - {\it ADAM}: Trajectory of Drosophila model}
      Temporal evolution of given initial state until steady state is reached.
	\begin{figure}[htb]
		\centering
		\includegraphics[width=0.95\textwidth]{DroTraj.jpg}
		%\includegraphics{DroTraj.jpg}
		\caption{{\it ADAM}: Trajectory of Drosophila model}
		\label{fig:traj}
	\end{figure}
	
	


	\subsection*{Figure 3 - Runtime of steady state calculations of several logical models from \cite{GINsimRepo}. Executed on a 2.7 GHz computer.}
	      Runtime of steady state calculations of several logical models from \cite{GINsimRepo}. Executed on a 2.7 GHz computer.
	
		\begin{figure}[htb]
			\centering
			%\includegraphics{GINSimChart.png}
			\includegraphics[width=0.7\textwidth]{GINSimChart.png}
			\caption{Runtime of steady state calculations of several logical models from
				\cite{GINsimRepo}. Executed on a 2.7 GHz computer.}
			\label{fig:chart}
		\end{figure}
	
	
	
		\subsection*{Figure 4 - (left) Wiring diagram: static relationship between variables (right) Phase space: temporal evolution of the system.}
		(left) Wiring diagram: static relationship between variables (right) Phase space: temporal evolution of the system.
		
		\begin{figure}[ht]
			\centering
			\includegraphics[scale=0.55]{exampleWD.jpg}
			\includegraphics[scale=0.55]{exampleSS.jpg}
			\caption{(left)
				Wiring diagram: static relationship between variables
				(right)
				Phase space: temporal evolution of the system
			}
			\label{fig:ex}
		\end{figure}
		
		
%%%%%%%%%%%%%%%%%%%%%%%%%%%%%%%%%%%	
	

	\subsection*{Table 1 - Correspondence of genes and variable names}
		\ul{Genes and proteins in} \cite{AO} \ul{and their corresponding variable names $x_1, \ldots, x_{60}$. }
		\par \mbox{}
	    \par
	    \mbox{
		\begin{tabular}{|c|c|c|c|c|c|c|c|c|c|c|c|c|c|c|c|}
		  \hline
		  \multirow{2}{*}{Cell 1}&SLP & wg & WG & en & EN & hh & HH & ptc & PTC & PH & SMO & ci & CI & CIA &  CIR \\
			&x1  & x2 &  x3 &  x4 &  x5 & x6 & x7 & x8 &   x9 &   x10 & x11 & x12 & x13 & x14 & x15\\
		  \hline

		  \multirow{2}{*}{Cell 2}&SLP & wg & WG & en & EN & hh & HH & ptc & PTC & PH & SMO & ci & CI & CIA &  CIR \\
		 &x16 & x17 & x18 & x19 & x20 & x21 & x22 & x23 & x24 & x25 & x26 & x27 & x28 & x29 & x30\\
		  \hline

		  \multirow{2}{*}{Cell 3}&SLP & wg & WG & en & EN & hh & HH & ptc & PTC & PH & SMO & ci & CI & CIA &  CIR \\
		 &x31 & x32 & x33 & x34 & x35 & x36 & x37 & x38 & x39 & x40 & x41 & x42 & x43 & x44 & x45\\
		  \hline

		  \multirow{2}{*}{Cell 4}&SLP & wg & WG & en & EN & hh & HH & ptc & PTC & PH & SMO & ci & CI & CIA &  CIR \\
		&x46 & x47 & x48 & x49 & x50 & x51 & x52 & x53 & x54 & x55 & x56 & x57 & x58 & x59 & x60\\
		  \hline
		\end{tabular}
		\label{tab:AOvariables}
	}
	


	
	
\subsection*{Table 2 - Genes and proteins present in steady state}
	Genes and proteins present in steady state corresponding to binary string \eqref{equ:ss}.
	\par \mbox{}
    \par
    \mbox{
		\begin{tabular}{|l|l|}
			\hline
			compartment 1 & en,	EN,	hh,	HH,	SMO\\
			\hline
			compartment 2 &ptc, PTC, PH, SMO, ci, CI, CIA\\
			\hline
			compartment 3 &SLP, PTC, ci, CI, CIR\\
			\hline
			compartment 4 &SLP, wg, WG, ptc, PTC, PH, SMO, ci, CI, CIA\\
			\hline
		\end{tabular}
		\label{table:ss}
	}
	
	\subsection*{Table 3 - Genes and proteins present in initial state}
		\ul{Genes and proteins present in initial state corresponding to binary string} \eqref{equ:ini}.
		\par \mbox{}
	    \par
	    \mbox{
			\begin{tabular}{|l|l|}
				\hline
				compartment 1 & en,	hh\\
				\hline
				compartment 2 &ptc,  ci \\
				\hline
				compartment 3 &SLP, ptc, ci\\
				\hline
				compartment 4 &SLP, wg, ptc, ci\\
				\hline
			\end{tabular}
			\label{table:ini}
		}	
	%0 0 0 1 0 1 0 0 0 0 0 0 0 0 0
	%0 0 0 0 0 0 0 1 0 0 0 1 0 0 0
	%1 0 0 0 0 0 0 1 0 0 0 1 0 0 0
	%1 1 0 0 0 0 0 1 0 0 0 1 0 0 0
	
	\subsection*{Table 4 - Software Comparison}
		\ul{Comparison of different software tools regarding attractor analysis:
		$\ddagger$ less than 1 second
	  		on published gene regulatory networks with up to 72 variables; 
		$\diamond$
	  		only for short limit cycles;
		$ \dagger$ heuristic methods are available for larger networks; 
			$\circ$ installation necessary, available for common operating systems.} \par \mbox{}
		    \par
		    \mbox{
	\begin{tabular}{|l||c|c|c|c|}
					\hline

					&  Steady State & Limit Cycle 	&{Input}	& {System } \\
					& Analysis & Analysis & Format & Requirements\\

					\hline
					\hline

					\multirow{2}{*}{ADAM} & \multirow{2}{*}{Yes$^\ddagger$}& \multirow{2}{*}{Yes$^\diamond$} & Boolean (or polynomial) functions &\multirow{2}{*}{None, web based} \\
						&&&Logical Models (GINsim) &\\

					\hline

					\multirow{2}{*}{GINsim}& \multirow{2}{*}{Yes$^\ddagger$} &
			    For small& Parameters (non-zero truth tables) & \multirow{2}{*}{Java virtual machine$^\circ$}\\
						&&models&Logical Model &\\
					\hline

					{BoolNet R } & For small $^ \dagger$  & For small $^ \dagger$ & \multirow{2}{*}{Boolean functions} & \multirow{2}{*}{R statistics software}\\
					package&models&models&&\\
					\hline


					\multirow{2}{*}{DDLab} & For small  & For small  & \multirow{2}{*}{Logical tables} & \multirow{2}{*}{$^\circ$}\\
						&models&models&&\\
					\hline


					BN/PBN Matlab  & {For small } & {For small  } & \multirow{2}{*}{Logical tables} & \multirow{2}{*}{Matlab}\\
					Toolbox & models& models& &\\
					\hline

				\end{tabular} 

			  \label{table:comparison}
	}



	\subsection*{Table 5 - Multi-valued models}
		Updates for variable $x_2$ in a logical model, where $x_2$ depends on $x_1$ and itself. The states $0$ and $1$ represent absent and present for the Boolean variable $x_1$; $0$, $1$, and $2$ represent low, medium, and high for the multi-valued variable $x_2$. The last row is introduced in the polynomial dynamical system such that all variables are defined over $\mathbb F_3$. The extra states $(2,0), (2,1), (2,2)$ in the state space should be ignored when interpreting the dynamics. \par \mbox{}
		\par
		\mbox{
			\begin{tabular}{|c||c|c|c|}
				\hline
				 		next state of $x_2$& low $x_2$ & medium $x_2$  & high $x_2$  \\

				\hline \hline
					$x_1$ absent & low $x_2$ & medium $x_2$ & high $x_2$ \\
				\hline 
					$x_1$ present & medium $x_2$ & high $x_2$  & high $x_2$ \\
				\hline
					extension $x_1$ present & medium $x_2$ & high $x_2$  & high $x_2$ \\
				\hline 
			\end{tabular}
			\label{table:extraStates}
		}


\end{bmcformat}
\end{document}






