\documentclass[11pt]{amsart}
\usepackage{graphicx,xspace}
\usepackage{fullpage}
\usepackage{hyperref}
\usepackage{amsmath}
\usepackage{amsfonts}
\usepackage{verbatim}

\newenvironment{definition}[1][Definition]{\begin{trivlist}
\item[\hskip \labelsep {\bfseries #1}]}{\end{trivlist}}
\newenvironment{example}[1][Example]{\begin{trivlist}
\item[\hskip \labelsep {\bfseries #1}]}{\end{trivlist}}



\title{Algebraic Framework for Discrete Models in Systems Biology} 
\author{Franziska Hinkelmann$^{1,2}$%
       %\email{Franziska Hinkelmann - fhinkel@vt.edu}%
      \and
         Madison Brandon$^{3,\dagger}$%
         %\email{Madison Brandon - mbrando1@utk.edu}
		\and
         Bonny Guang$^{4,\dagger}$%
         %\email{Bonny Guang - bonny.guang@gmail.com}
		\and
	     Rustin McNeill$^{5,\dagger}$%
	      %\email{Rustin McNeill - rcmcneil@uncg.edu}
		\and
         Grigoriy Blekherman$^1$%
          %\email{Grigoriy Blekherman - grrigg@vbi.vt.edu}	
		\and
         Alan Veliz-Cuba$^{1,2}$%
         %\email{Alan Veliz-Cuba - alanavc@vt.edu}
       and 
         Reinhard Laubenbacher$^{1,2}$%
         %\email{Reinhard Laubenbacher - reinhard@vbi.vt.edu}%
      }
      

%%%%%%%%%%%%%%%%%%%%%%%%%%%%%%%%%%%%%%%%%%%%%%
%%                                          %%
%% Enter the authors' addresses here        %%
%%                                          %%
%%%%%%%%%%%%%%%%%%%%%%%%%%%%%%%%%%%%%%%%%%%%%%

\address{%
    $^1$Virginia Bioinformatics Institute, Blacksburg, VA 24061-0123, USA\\
    $^2$Virginia Polytechnic Institute and State University, Blacksburg, VA 24061-0123, USA\\
    $^3$University of Tennessee - Knoxville, Knoxville, TN 37996-2513, USA\\
    $^4$Harvey Mudd College, Claremont, CA 91711-5901, USA\\
    $^5$University of North Carolina - Greensboro, Greensboro, NC 27402-6170, USA\\
    $^\dagger$These authors contributed equally
}%
\begin{document}

\begin{abstract}
Many biological systems are modeled qualitatively with discrete models,
such as (probabilistic) Boolean networks, logical models, Petri nets,
and agent-based models.  Simulation is a common practice for analyzing
discrete models, but many systems are far too large to capture all the
relevant dynamical features through simulation alone.  
We convert discrete models into algebraic models and apply tools from computational algebra to analyze their dynamics.  The key feature of biological systems that is exploited by our algorithms is their sparsity: while the number of nodes in a biological network may be quite large, each node is affected only by a small number of other nodes. In our experience with models arising in systems biology, this structure leads to fast computations when using algebraic models, and thus efficient analysis.  
All algorithms and methods are available in our package Analysis of Dynamic
Algebraic Models (ADAM), a user friendly web-based tool that allows for fast
analysis of large models, without requiring understanding of the underlying
mathematics or any software installation. 
\end{abstract}

\maketitle
\end{document}
