\documentclass{beamer}

\title{June 11th, 2010 - Boolean Networks}
\author{Bonny Guang, Madison Brandon, Rustin McNeill}
\date{June 11th, 2010}

\begin{document}

\maketitle

\begin{frame}
 \frametitle{Goals}

What we want to accomplish during the next week\\
\begin{itemize}
	\item{Continue writing code in Macaulay2}
	\item{Become familiar with each of our biological systems}

What we want to accomplish over the next eight weeks \\
\begin{itemize}
	\item{implement Groebner Bases in order to analyze boolean logic models}
	\item{Develop mathematical algorithms to analyze dynamics of large biological networks}
	\item{Be able to identify key dynamical features in systems with large numbers of states}
	\item{Write software using Macaulay2 to be incorporated into Polynome}
\end{itemize}

\end{frame}

\begin{frame}
 \frametitle{Biological Systems}

Yeast cell cycle\\
\begin{itemize}
\item{Logical modelling of the core engine controlling the budding yeast cell cycle}
\item{Most common form of production in yeast is budding}
\item{Model of what activates and inhibits yeast cell production}
\end{itemize}

T lymphocytes diffentiation\\
\begin{itemize}
\item{Are a type of white blood cell and part of the vertebrate immune system}
\item{Activation or inhibition of proteins can cause various responses}
\end{itemize}

Network controlling ErbB2 activation\\
\begin{itemize}
\item{A model displaying what activates and inhibits ErbB2 production}
\item{Overexpression of the protein ErbB2 has been observed in nearly 30% of cancer patients}
\item{Being able to analyze the system can lead to a greater understanding of why ErbB2 is overexpressed} 
\end{itemize}
\end{frame}



\end{document}
